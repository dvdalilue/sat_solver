\cleardoublepage
\phantomsection
\addcontentsline{toc}{chapter}{Abstract}
\thispagestyle{pagebottom}
\noindent\textbf{\Huge Abstract}

\vspace{2em}

The \sat problem intends to determine if there is a possible combination of values which satisfies a boolean proposition. This problem was the first in being prove to be NP-complete, there is no algorithm such gives a solution to this problem in an efficient way, i.e. polynomial time. On the other hand, we have the CSP problems, these checks if a finite number of constraints, with a specific number of variables, can be satisfy. This problems can be interpreted and reduced to \sat. Given this, we are focusing on this famous problem, since it has the power to embrace different problems, translating the conditions into logical formulas.

Currently, and since a few years, \sat solvers implements two kinds of algorithm, which are these: conflict-driven clause learning algorithm, based in DPLL, and stochastics local searchs. Both have many implementations, some are famous \sat solvers, other use a paralell strategy like \textit{ManySAT}. These algorithms need to rewrite the propositional formula into its conjunctive normal form (CNF), allowing to split the formula in clauses which ease the tests of satisfiability. Other formula representations could bring a different approach, and in some cases, they perform better. For example, the binary decision diagrams (BDD).

All this give us the motivation to use a different representation of boolean propositions, letting us have a new approach to \sat. The \textit{Zhegalkin} polynomial allow us to have a new point of view and its shape brings up the possibility to use parallelism in the implementation, using GPU resources. This representation allow us to convert a boolean proposition into a polynomial, where the variables take binary values. Taking advantage of its structure, similar to normal polynomials, it is possible to use parallel method with little variations, aiming to find a new solution to \sat.

\newpage