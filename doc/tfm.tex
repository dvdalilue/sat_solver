\documentclass[12pt,letterpaper,titlepage,oneside,openright]{book}

\usepackage[utf8]{inputenc}
\usepackage[spanish]{babel}
\usepackage[svgnames]{xcolor}
\usepackage[T1]{fontenc}
% \usepackage[titletoc]{appendix}
\usepackage{amsmath,amsfonts,amsthm}
\usepackage{enumitem}
\usepackage{hyperref}
\usepackage{graphicx}
\usepackage{pdfpages}
\usepackage{lettrine}
\usepackage{listings}
\usepackage{titlesec}
\usepackage{epigraph}
\usepackage{lipsum}
% \usepackage{setspace}
% \usepackage{multirow}
% \usepackage{fancyhdr}
% \usepackage{relsize}
\usepackage{xspace}
\usepackage{cite}

\makeatletter
\renewcommand{\@chapapp}{}% Not necessary...
\newenvironment{chapquote}[2][2em]
  {\setlength{\@tempdima}{#1}%
   \def\chapquote@author{#2}%
   \parshape 1 \@tempdima \dimexpr\textwidth-2\@tempdima\relax%
   \itshape}
  {\par\normalfont\hfill--\ \chapquote@author\hspace*{\@tempdima}\par\bigskip}
\makeatother

\begin{document}

\frontmatter

\begin{titlepage}
    \begin{center}
        \includegraphics[width=0.15\textwidth]{logo.png}\\
        {\large UNIVERSIDAD POLITÉCNICA DE MADRID}\\
        ESCUELA TÉCNICA SUPERIOR DE INGENIEROS INFORMÁTICOS\\
        DEPARTAMENTO DE LENGUAJES Y SISTEMAS INFORMÁTICOS E INGENIERÍA DE SOFTWARE

        \vspace{6em}

        \textbf{TITULO QUE TAL QUE PIN QUE PAO}

        \vspace{6em}

        Por:\\
        David Alejandro Lilue Borrero\\

        \vspace{6em}

        TRABAJO FINAL DE MASTER\\
        Presentado ante la ilustre Universidad Politécnica de Madrid\\
        como requisito para optar al título de Máster en\\
        Software y Sistemas\\
        
        \vfill
        Madrid, Junio de 2018
    \end{center}
\end{titlepage}

\newpage

\cleardoublepage
\phantomsection
\addcontentsline{toc}{chapter}{Dedicatoria}
\begin{flushright}
Dedicalo a alguien o algo.
\end{flushright}

\newpage

\cleardoublepage
\phantomsection
\addcontentsline{toc}{chapter}{Agradecimientos}
\begin{flushleft}
Agradezco a un kiwi que vuela en un avión de papel,\\
deja en su camino una estala con forma de espiral.\\
Aquella posee una proporción casi divina.
\end{flushleft}

\newpage

\tableofcontents

\newpage

\mainmatter

\chapter{Introducción}

\begin{chapquote}{Miguel de Cervantes, \textit{Don Quijote de la Mancha}}
Sola una cosa tiene mala el sueño, según he oído decir, y es que se parece a la muerte, pues de un dormido a un muerto hay muy poca diferencia.
\end{chapquote}

\lipsum[1-2]

\chapter{Marco teórico}

\lipsum[1-2]

\end{document}