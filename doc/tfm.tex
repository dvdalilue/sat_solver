\documentclass[12pt,letterpaper,titlepage,oneside,openright]{book}

\usepackage[utf8]{inputenc}
\usepackage[spanish]{babel}
\usepackage[svgnames]{xcolor}
\usepackage[T1]{fontenc}
% \usepackage[titletoc]{appendix}
\usepackage{amsmath,amsfonts,amsthm}
\usepackage{enumitem}
\usepackage{hyperref}
\usepackage{graphicx}
\usepackage{pdfpages}
\usepackage{lettrine}
\usepackage{listings}
\usepackage{titlesec}
\usepackage{epigraph}
\usepackage{lipsum}
% \usepackage{setspace}
% \usepackage{multirow}
% \usepackage{fancyhdr}
% \usepackage{relsize}
\usepackage{xspace}
\usepackage{cite}

\makeatletter
\renewcommand{\@chapapp}{}% Not necessary...
\newenvironment{chapquote}[2][2em]
  {\setlength{\@tempdima}{#1}%
   \def\chapquote@author{#2}%
   \parshape 1 \@tempdima \dimexpr\textwidth-2\@tempdima\relax%
   \itshape}
  {\par\vspace{.5em}\normalfont\hfill--\ \chapquote@author\hspace*{\@tempdima}\par\bigskip}
\makeatother

\begin{document}

\frontmatter

\begin{titlepage}
    \begin{center}
        \includegraphics[width=0.15\textwidth]{logo.png}\\
        {\large UNIVERSIDAD POLITÉCNICA DE MADRID}\\
        ESCUELA TÉCNICA SUPERIOR DE INGENIEROS INFORMÁTICOS\\
        DEPARTAMENTO DE LENGUAJES Y SISTEMAS INFORMÁTICOS E INGENIERÍA DE SOFTWARE

        \vspace{6em}

        \textbf{TITULO QUE TAL QUE PIN QUE PAO}

        \vspace{6em}

        Por:\\
        David Alejandro Lilue Borrero\\

        \vspace{6em}

        TRABAJO FINAL DE MASTER\\
        Presentado ante la ilustre Universidad Politécnica de Madrid\\
        como requisito para optar al título de Máster en\\
        Software y Sistemas\\
        
        \vfill
        Madrid, Junio de 2018
    \end{center}
\end{titlepage}

\newpage

\cleardoublepage
\phantomsection
\addcontentsline{toc}{chapter}{Dedicatoria}
\begin{flushright}
Para todas esas personas\\
que han hecho sentir a un desconocido como en casa,\\
donde la curiosidad y tolerancia superan los miedos,\\
y de alguna forma, mágica, nace la hermandad.
\end{flushright}

\newpage

\cleardoublepage
\phantomsection
\addcontentsline{toc}{chapter}{Agradecimientos}
\begin{flushleft}
Agradezco a ese avión de papel\\
que le ha permitido volar al kiwi\\
dejando en su camino una estala con forma de espiral,\\
aquella posee una proporción casi divina; de oro.
\end{flushleft}

\newpage

\tableofcontents

\newpage

\mainmatter

\chapter{Introducción}

\begin{chapquote}{Miguel de Cervantes, \textit{Don Quijote de la Mancha}}
Sola una cosa tiene mala el sueño, según he oído decir, y es que se parece a la muerte, pues de un dormido a un muerto hay muy poca diferencia.
\end{chapquote}

\lipsum[1-2]

\chapter{Marco teórico}

\begin{chapquote}{Rudyard Kipling, \textit{El gato que caminaba solo}}
Eres el Gato que camina solo y a quien no le importa estar aquí o allá. No eres un amigo ni
un servidor. Tú mismo lo has dicho. Márchate y camina solo por cualquier lugar.
\end{chapquote}

\lipsum[1-2]

\chapter{Marco metodolócico}

\begin{chapquote}{Andréi Tarkovski, \textit{Solaris}}
Cuando hace viento, es fácil confundir el vaivén de un arbusto lloroso con una criatura viviente.
\end{chapquote}

\lipsum[1-2]

\chapter{Motivación}

\begin{chapquote}{Emile Michel Cioran, \textit{Silogismos de la amargura}}
Sólo los espíritus agrietados poseen aberturas al más allá.
\end{chapquote}

\lipsum[1-2]

\chapter{Prototipo funcional}

\begin{chapquote}{Platón, \textit{Alegoría de la cueva}}
\textsc{Socrates}: Imagine this: People live under the earth in a cavelike dwelling. ... The people have been in this dwelling since childhood, shackled by the legs and neck. ... Some light, namely from a fire that casts its glow toward them from behind them, ... Between the fire and those who are shackled there runs a walkway at a certain height. ... a low wall has been built the length of the walkway, like the low curtain that puppeteers put up, ... along this low wall people are carrying all sorts of things that reach up higher than the wall.

\vspace{1em}

\textsc{Glaucon}: ... these are unusual prisoners.

\textsc{Socrates}: They are very much like us humans...
\end{chapquote}

\lipsum[1-2]

\chapter{Prototipo secuencial}

\begin{chapquote}{Radiohead, \textit{Lotus Flower}}
\noindent Slowly we unfurl\\
As lotus flowers\\
'Cause all I want is the moon upon a stick\\
I dance around the pit\\
The darkness is beneath\\
I can't kick your habit\\
Just to feed your fast ballooning head\\
Listen to your heart
\end{chapquote}

\lipsum[1-2]

\end{document}